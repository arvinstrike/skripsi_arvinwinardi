\onehalfspacing
% \begin{titlepage}
    \begin{center}    
    
        % \vspace*{1.0cm}
        % judul thesis harus dalam 14pt Times New Roman

        \begin{minipage}{\textwidth}
            \centering
            \bo{\Judul} 
        \end{minipage}
        

        
        \vspace*{0.5cm}
        \begin{figure}
            \begin{center}
                % \includegraphics[width=2.5cm]{_internals/makara.eps}
                \includegraphics[width=5cm]{assets/pics/logo_UMN_clean.png}
            \end{center}
        \end{figure}    
        % \vspace*{0.5cm}        % harus dalam 14pt Times New Roman
        \MakeUppercase{ \bo{\Type} }
        \vspace*{1cm}
               
        
        Diajukan sebagai salah satu syarat untuk memperoleh\\
        Gelar Sarjana Komputer (S.Kom.) \\[1cm]
        % penulis dan npm
        \MakeUppercase{ \bo{\penulis}} \\
        \bo{\nim} \\

        \vfill

        % informasi mengenai fakultas dan program studi
        \bo{
        	PROGRAM STUDI \Program \\
        	FAKULTAS \Fakultas\\
        	UNIVERSITAS MULTIMEDIA NUSANTARA\\
        	TANGERANG \\
        	\tahun
        }
    \end{center}
% \end{titlepage}

\newpage
