%-----------------------------------------------------------------------------%
\chapter*{\Judul}
%-----------------------------------------------------------------------------%
\singlespacing
\begin{center}
    
    \vspace{-4em}
    
    \penulis
    
	\bigskip
    
    \textbf{ABSTRAK}
    
\end{center}

% \chapter*{Abstrak}

\vspace*{0.2cm}
{
	\setlength{\parindent}{0pt}

	\bigskip
	\bigskip

    \noindent
    Kemajuan teknologi \textit{deepfake} telah menimbulkan tantangan signifikan dalam menjaga integritas informasi digital. Penelitian ini mengusulkan sistem klasifikasi \textit{deepfake} pada gambar wajah menggunakan pendekatan \textit{ensemble deep learning} dengan metode \textit{weighted averaging}. Empat model individu digunakan dalam ansambel: Custom CNN, ResNet50, Xception, dan EfficientNet-B4. Dataset yang digunakan adalah \textit{140k Real and Fake Faces} dari Kaggle, dengan partisi data pelatihan, validasi, dan pengujian sebesar 100.000, 20.000, dan 20.000 gambar. Setiap model dilatih secara independen dan dievaluasi menggunakan metrik Akurasi, Presisi, Recall, F1-Score. Hasil eksperimen menunjukkan bahwa model ansambel menghasilkan akurasi sebesar 96.87\%, lebih tinggi dibandingkan model individual terbaik (Xception, 95.83\%). Evaluasi \textit{cross-dataset} menggunakan \textit{DeepFakeFace} menunjukkan bahwa meskipun akurasi menurun menjadi 50\%, ansambel tetap menunjukkan kinerja generalisasi yang lebih baik dibandingkan model tunggal. Penelitian ini menunjukkan bahwa pendekatan \textit{ensemble} dengan arsitektur yang beragam dapat meningkatkan akurasi dan keandalan sistem deteksi \textit{deepfake}.
	\bigskip
% Kata kunci urut abjad

	\textbf{Kata kunci}: 	\textit{Deepfake, Image Detection, Ensemble Learning, Weighted Averaging, CNN}
}

\onehalfspacing